\documentclass[twocolumn]{aastex61}

\usepackage[utf8]{inputenc}
\usepackage{url}
\usepackage{listings}
\usepackage{xspace}
\usepackage{natbib}  % Requires natbib.sty, available from http://ads.harvard.edu/pubs/bibtex/astronat/

\newcommand{\pyspeckit}{\texttt{pyspeckit}\xspace}
\newcommand{\astropy}{\texttt{astropy}\xspace}
\begin{document}

\title{pyspeckit}
\authorrunning{Ginsburg et al}

\author[0000-0001-6431-9633]{Adam Ginsburg}
\nraojansky


\begin{abstract}
Pyspeckit is a tool and library for spectroscopic analysis. 
\end{abstract}


\section{Introduction}
Spectroscopy is an important tool for astronomy.

\section{Basic structure}
The central object in \pyspeckit is a \texttt{Spectrum} object, which has
associated \texttt{data}, \texttt{error}, and \texttt{xarr}, the latter of
which represents the spectroscopic axis.  A \texttt{Spectrum} object has
several attributes that are themselves sophisticated classes that can be called
as functions: the \texttt{plotter}, the fitter \texttt{specfit}, and the
continuum fitter \texttt{baseline}.

\subsection{Plotter}
The plotter is a basic line plotter.  See the GUI section (\S \ref{sec:gui})
for more details.

\subsection{Fitter}
The fitting tool in \pyspeckit is the \texttt{Spectrum.specfit} object.
This object is a class that is created for every \texttt{Spectrum} object.
The fitter can be used with any of the models included in the model
registry.

To fit a profile to a spectrum, several optimizers are available.  Two
implementations of the Levenberg-Marquardt optimization method are provided,
\texttt{mpfit}\footnote{Originally implemented by Craig Markwardt
\url{https://www.physics.wisc.edu/~craigm/idl/fitqa.html} and ported to python
by Mark Rivers and then Sergei Koposov.  The version in pyspeckit has been
updated somewhat from Koposov's version.} and
\texttt{lmfit}\footnote{\url{https://lmfit.github.io/lmfit-py/},
\url{http://dx.doi.org/10.5281/zenodo.11813}}.  Wrappers of
\texttt{pymc}\footnote{\url{https://pymc-devs.github.io/pymc/}} and
\texttt{emcee}\footnote{\url{http://dfm.io/emcee/current/},
\citet{Foreman-Mackey2013a}} are also available, though these tools are better
for parameter error analysis than for optimization.

The basic approach to fit a spectrum is to assign an error to each pixel in the
\texttt{data} array by providing a value in the \texttt{sp.error} array.  These
values are assumed to be $1-\sigma$ Gaussian errors on the data.

One convenience provided by \pyspeckit is also a potential `gotcha': if no
error is provided, the first time a spectrum is fit, the error will be
automatically set by computing the standard deviation of the fit residuals.
Fitting the same spectrum twice may therefore result in different parameter
errors, but it should never change the fitted parameter values.


\section{Graphical Design Choices}
\label{sec:gui}
\subsection{GUI development}
Many astronomers are familiar with IRAF's \texttt{splot} tool, which is useful
for fitting Gaussian profiles to spectral lines.  It used keyboard interaction
to specify the fitting region and guesses for fitting the line profile, but for
most users, gave them access to those results \emph{only} through the GUI.

\texttt{pyspeckit}'s fitting GUI was built to match \texttt{splot}'s
functionality, but with addition means of interacting with the fitter.  In
\texttt{splot}, reproducing any given fit is challenging, since subtle changes
in the cursor position can significantly change the fit result.  In \pyspeckit,
it is possible to record the results of fits programatically and re-fit using
those same results.

The GUI was built using \texttt{matplotlib}'s canvas interaction tools.  These
tools are limited to GUI capabilities that are compatible with all platforms
(e.g., Qt, Tk, Gtk) and therefore exclude some of the more sophisticated fitting
tools found in other software (e.g., \texttt{glue}).

\subsection{Plotting}
Plotting in \pyspeckit is meant to provide the shortest path to
publication-quality figures.  The default plotting mode uses histogram-style
line plots and labels axes with \LaTeX-formatted versions of units.

When the plotter is active and a model is fit, the model parameters are
displayed with \LaTeX formatting.  The errors on the parameters, if available,
are also shown, and these errors are used to decide on how many significant
figures to display.

\section{\astropy integration}
Development of \pyspeckit began several years before \astropy began.  Several
features were therefore implemented that were later replaced by similar
\astropy features, in particular the unit system.  Unit conversions in
\pyspeckit are now (as of October 2015) done entirely using the \astropy
system.

\section{Models}
While many of \pyspeckit's internal functions are likely to replaced by
\astropy tools and affiliated packages, the models in \pyspeckit are likely to
remain useful indefinitely.  The model library in \pyspeckit includes some of
the most useful general functions (e.g., Gaussian, Lorentzian, and Voigt
profiles) and a wide range of specific model types.

In radio and millimeter astromomy, there are several molecular line groups that
consist of several Gaussian profiles separated by a fixed frequency offset.
These hyperfine line groups are often unique probes of physical parameters
because in many cases the ``main'' line can become optically thick, while the
`satellite' lines remain thin, meaning that the line optical depth can be
measured with a single spectrum.  \pyspeckit provides the \texttt{hyperfine}
model class to handle this class of molecular line transitions, and it includes
several molecular species specific implementations (HCN, N$_2$H$^+$, NH$_3$,
H$_2$CO).

\bibliographystyle{aasjournal}
\bibliography{extracted}


\end{document}
